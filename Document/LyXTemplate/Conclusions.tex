
\lhead[\chaptername~\thechapter]{\rightmark}

\rhead[\leftmark]{}

\lfoot[\thepage]{}

\cfoot{}

\rfoot[]{\thepage}

\chapter{Conclusions }

This thesis assesses the ability of detecting the performance of the
Ericsson's software products by applying the Markov switching autoregressive
model, and the E-divisive method which is an approach in the non-parametric
analysis. The simulated data with known state are used for the comparison
between both methods. The results from a real data are presented with
interpretations and discussions. 

For the Markov switching model, the number of states and the number
of switching coefficients in the model were determined and chosen
by examining the BIC, along with model outputs and plots. The findings
from the simulated data revealed that the Markov switching model is
able to discover the switches between states rather well despite some
false alarms and missed detections. On the contrary, the E-divisive
method is less powerful compare to the Markov switching model. This
method is efficient if data have an obvious pattern of shifting in
the time series. 

Moreover, an implemented function of the state prediction appear to
functionally work well when investigating on the simulated data. The
accuracy from both simulated datasets were high. From performing both
methods in the simulated data and the real data, the results from
the Markov switching model was considered to be the representative
for the analysis. The E-divisive method was generally used as a guideline
and also a confirmation for a change in the CPU state.

Evaluating the obtained results is rather difficult and complicated,
mostly due to a lack of annotations or label of the state of the CPU.
This is a common situation to an unsupervised learning problem where
the ground truth is not often available. Besides, another difficulty
of the Markov switching model usage is that a state inference is required.
Since the Markov switching model assumed a latent state, a sensible
inference is needed in order to get a final and meaningful results
that can be further used.

\section{Further work}

The Markov switching model gave quite a promising result but several
improvements could also be done to increase the robustness of the
analysis. 

For future work, using more observations in data in the analysis is
recommended as it will make obtained results be more reliable. Additional
information will decrease an uncertainty in the data. 

Another future extension is to consider on other predictor variables
or interaction terms which can also affect the CPU utilization. As
the assumption of the distribution of residuals was not entirely fulfill,
significant details that are used to explained the CPU utilization
might not all be caught by the model. Other distribution could also
be made apart from a normal distribution. 

Furthermore, there are still some performance metrics in QA Capacity
area which have not been taken into account in the thesis such as
a memory usage and a latency. Further work could also be extended
to analyze these metrics.

Finally, in the future if some test cases have been labeled with state
from a domain expert, a semi-supervised learning algorithm, a technique
that falls between supervised learning and unsupervised learning,
could also be implemented. Training a model with a large amount of
unlabeled data and a small amount of labeled data could considerable
improving an accuracy of the model. The semi-supervised technique
is a great practical use as labeling all data is a very time-consuming
operation. 

\begin{comment}
This line indicates the amount of time the CPU spent in each of several
states, displayed as percentages. The numbers reflect the percentage
of the time since the last screen update that the CPU spent in each
state. 

Idle: Nothing to do
\end{comment}


