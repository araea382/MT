
\chapter*{Abstract}

\addcontentsline{toc}{chapter}{Abstract} 

The core idea of this thesis is to reduce workload of manual inspection
when the performance analysis of an updated software is required.
CPU utilization, which is one of the essential factors for evaluating
the performance, is analyzed. The purpose of this work is to apply
machine learning techniques that are suitable for detecting the state
of the CPU utilization, and also any changes in the test environment
that affects the CPU utilization. The detection relies on a Markov
switching model to identify structural changes, which are assumed
to follow an unobserved Markov chain, in the time series data. A historical
behavior of the data can be described by a first-order autoregression.
Then, the Markov switching model becomes a Markov switching autoregressive
model. Another approach based on a non-parametric analysis, a distribution-free
method that requires fewer assumptions, called an E-divisive method
is purposed. This method uses a hierarchical clustering algorithm
to detect multiple change point locations in the time series data.
As the data used in this analysis does not contain any ground truth,
the evaluation of the methods is analyzed by generating simulated
datasets with known states. Besides, these simulated datasets are
used for studying and comparing between the Markov switching model
and the E-divisive method. Results show that the former method is
preferable because of its better performance and it is more efficient
in detecting changes. Some information about the state of the CPU
utilization are also obtained from performing the Markov switching
model. The E-divisive method is proved to have less power in detecting
changes and has a higher rate of missed detections. The results from
applying the Markov switching autoregressive model to the real data
are presented with interpretations and discussions. 

\textbf{Keywords:} Markov switching model, Non-parametric analysis,
CPU utilization
