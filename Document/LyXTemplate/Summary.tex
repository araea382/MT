
\chapter*{Abstract}

\addcontentsline{toc}{chapter}{Abstract} 

The purpose of this work is to apply machine learning techniques that
are suitable for detecting any changes in the state of the CPU, and
also in the test environment that effects the CPU utilization. The
CPU utilization which is one of the performance metrics of the software
is analyzed. The detection relies on a time series model, based on
a Markov switching model, to identify the changes. A historical behavior
of the data can be described by a first-order autoregression. The
Markov switching model then becomes a Markov switching autoregressive
model. Another approach based on a non-parametric analysis, a distribution-free
method that requires fewer assumptions, called an E-divisive method
is purposed. The method is used to detect any abrupt change point
locations in the time series data. Since the given data does not contain
any ground truth about the state of the CPU, the evaluation of the
methods cannot be made. A comparison between the Markov switching
model and the E-divisive method are studied using simulated datasets
with a labeled data. Results show that the former method is preferable
because of its efficiency and providing a better performance. 

\textbf{Keywords:} Markov switching model, Non-parametric analysis,
CPU utilization
