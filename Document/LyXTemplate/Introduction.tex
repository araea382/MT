
\lhead[\chaptername~\thechapter]{\rightmark}

\rhead[\leftmark]{}

\lfoot[\thepage]{}

\cfoot{}

\rfoot[]{\thepage}

\chapter{Introduction}

\section{Background \label{sec:Background}}

In this study, \emph{change point analysis} will be used to identify
changes over time in performance of Ericsson's software products.
Many test cases are executed for testing software packages in a simulation
environment. Before launching the software products to its customers,
the company needs to test and determine how each software package
performs. The performance of these software packages is evaluated
by considering on CPU utilization (percentages of CPU's cycle spent
on each process), memory usage, and latency. 

The Central Processing Unit (CPU) is an electronic component that
responsible for executing the commands from the 

Structural changes are often seen in time series data. This observable
behavior is highly appealing to statistical modelers who want to develop
a model which is well explained. A method to detect changes in time
series data when a time index is unknown is called \emph{change point
analysis} \citep{basseville1993detection}. The analysis discovers
the time point where the changes occur. Change point analysis can
be referred to different kinds of name such as breakpoint and turning
point. However, \emph{change-point} is the commonly used term for
the point in a time series where a change takes place. Another important
term used in this area is \emph{regime switch} which refers to persistent
changes in time series structure after the occurrence of change point
\citep{weskamp2010change}. Change point analysis has been studied
for several decades as it is a problem of interest in many applications
in which the characteristic of data is collected over time. A change
should be flagged as soon as it occurs in order to be properly dealt
with reducing any possible consequences \citep{sharkey2014nonparametric}.
Here are some examples.
\begin{itemize}
\item Medical condition monitoring: Evaluate the sleep quality of patients
based on their heart rate condition \citep{staudacher2005new}. 
\item Climate analysis: The temperature or climate variations are detected.
This method has gradually become important over the past few decades
due to the effects of the global warming and the increases in greenhouse
gas emissions \citep{reeves2007review,beaulieu2012change}. 
\item Quality control: Since industrial production is a continuous production
process, in mass production process, if the product controlled value
is not monitored and exceeds the tolerable level undetected, it could
lead to the loss of a whole production lot \citep{page1954continuous}. 
\item Other applications: Identifying fraud transaction \citep{bolton2002statistical},
detecting anomalies in the market price \citep{gu2013fast} and detecting
signal processing \citep{basseville1993detection} in streaming data
as well. 
\end{itemize}
In recent years, a method called hidden Markov model or Markov switching
model has become widely used for discovering change points in time
series. Both terms are accepted, usage varies with different fields
of study. Markov switching model uses a concept of a Markov chain
to model an underlying segmentation as different states and then specify
a distinct change of location. Hence, the method is able to identify
a switch in time series when change point occurs \citep{luong2012hidden}.
This method is used in almost all current systems in speech recognition
\citep{rabiner1989tutorial} and is found to be important in climatology
such as describing the state in the wind speed time series \citep{ailliot2012markov}
and in biology \citep{stanke2003gene} where protein coding genes
are predicted. Markov switching model has been extensively applied
in the field of economics and finance and has a large literature.
For example, business cycles can be seen as hidden states with seasonal
changes. The growth rate of gross domestic product (GDP) is modeled
as a switching process to uncover business cycle phases i.e., expansion
and recession. The fitting model can also be used to understand the
process where there is a transition between the economic state and
the duration of each period \citep{hamilton1989new}. In finance data,
time series of returns is modeled in order to investigate stock market
situation i.e., bull or bear market \citep{kim1998testing}. 

Markov switching model is one of the most well-known non linear time
series models. This model can be applied to various time series data
with dynamic behavior. The structural changes or regime shifts in
data implies that constant parameter settings in a time series model
might be insufficient to capture these behaviors and describe their
evolution. Markov switching model takes the presence of shifting regime
in time series into account and models multiple structures that can
explain these characteristics in different states at different time.
The shift between states or regimes comes from the switching mechanism
which is assumed to follow an unobserved Markov chain. Thus, the model
is able to capture more complex dynamic patterns and also identify
the change of locations and regime switch in time series. 

For the current Ericsson setting, each software package version running
through the test system is viewed as a time point in time series and
the performance of each software package is treated as an observed
value. It is proven that the observed values are not completely independent
of each other i.e., the performance of the current software package
depends on the performance from the prior version of the software
package. Therefore, additional dependencies are taken into consideration
by a first-order autoregression when modeling. The Markov switching
model becomes the Markov switching autoregressive model. This model
is applied to the given data in order to discover the changes in the
performance.

There are two approaches, a parametric and a non-parametric analysis,
for detecting the change point in the time series. The parametric
analysis benefits from assuming some knowledge of data distribution
and integrating it to the detection scheme. On the other hand, the
non-parametric analysis is more flexible in that there are no assumption
made about the distribution. It can, therefore, apply to a wider range
of applications and capture various kinds of changes \citep{sharkey2014nonparametric}.
The non-parametric analysis using hierarchical estimation techniques
based on a divisive algorithm is used. This method, which is called
an E-divisive, is designed to perform multiple change point analysis
while trying to make a few assumptions as possible. The E-divisive
method estimates change points by using a binary bisection approach
and a permutation test. The method also capable of estimating not
only univariate data but also multivariate data. 

In this study, the parametric analysis using the Markov switching
autoregressive model and the non-parametric analysis using the E-divisive
method are used for identifying change point locations in the time
series data. 

\section{Objective \label{sec:Objective}}

The core idea of this thesis is to reduce workload of manual inspection
when the performance analysis of an updated software package is required.
With an increased amount of generated data from numerous test cases,
the inspection becomes very tedious and inefficient to be done manually.
The main objective of this thesis is to implement a machine learning
algorithm that has an ability to learn from data in order to analyze
the performance of the software package. The algorithm will help indicate
whether the performance of the software package is in a degrading,
improving or steady state. It is also worth mentioning that the performance
of a particular software package can vary on different test environments.
The implemented algorithm should also be able to detect when the test
environment is altered. This thesis only focuses on the CPU utilization,
which is one of the three essential factors for evaluating the performance
of the upgraded software package.

To summarize, this thesis aims to:
\begin{itemize}
\item Detect the state of the CPU utilization (degrading, improving, or
steady state)
\item Detect whether there is any change in the test environment that effects
the CPU utilization
\end{itemize}
The thesis is structured as follow: Chapter 2 provides details and
descriptions of datasets used in the analysis. Chapter 3 presents
methodology. Results from the analysis along with tables and plots
are shown in Chapter 4. Chapter 5 discusses the outcomes and the obtained
results. Lastly, Chapter 6 contains conclusion and future work.


